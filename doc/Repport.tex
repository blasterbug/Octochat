\documentclass[a4paper]{article}
\usepackage[utf8]{inputenc}
\usepackage[frenchb]{babel}
\usepackage{ifpdf}
\usepackage{hyperref}
\title{Octochat \\ \textit{Discuter sur le réseau local sans serveur}}
\author{Alexis \textsc{Giraudet} \and Benjamin \textsc{Sientzoff}}
\date{\today}
\ifpdf
\hypersetup{
    pdfauthor={Alexis Giraudet, Benjamin Sientzoff},
    pdftitle={Octochat - Discuter sur le réseau local sans serveur},
}
\fi
\begin{document}
	% page de garde avec sommaire
	\maketitle
	\vspace{5cm}
	\tableofcontents
	\newpage % passer à la page suivante
	
	\section*{Introduction}
		\paragraph{}{
		Ce projet a été réalisé dans le cadre du cours \textit{Objet et développement d'applications}
		dans lequel M. \textsc{Richoux} nous a enseigné l'utilisation des \textit{Design Patterns}.
		L'ambition de ce projet ne s'arrête pas là, car nous souhaitons poursuite le développement de notre 
		programme. Le sujet de notre projet est la création d'un client de chat qui n'utilise pas de serveur
		principal comme c'est le cas pour ce genre d'application réseau. Le fonctionnement est détaillé plus loin.
		}
		\paragraph{}{
		Octochat, notre programme, est donc un client de chat qui n'a pas besoin de serveur pour fonctionner.
		Lancer le programme, choisissez un nom d'utilisateur est c'est parti.
		}
	
	\newpage
	
	\section{Utilisation et fonctionnement global}
	
		\paragraph{Boost}{
		Pour des questions de dépendances et pour faciliter le développement du programme, notamment pour ce qui est du réseau,
		nous avons choisi d'utiliser la librairie \textit{Boost}.
		L’utilisation de cette librairie nous permet également d'utiliser le système de compilation associé.
		}
	
		\subsection{Compilation du programme}
			\paragraph{Compilation}{
			Pour commencer, le projet à besoin d'un installation locale de \textit{boost}. Cette installation est faite
			lorsque le \textit{makefile} est appelé pour la première fois, mais une installation en bonne et due forme est
			préférable. Procédez comme suit.
			}
			\begin{verbatim}
				$ cd Octochat/
				$ ./bin/install-boost.sh
				$ ./bin/prepare-boost.sh
				$ make
			\end{verbatim}
			\paragraph{}{
			La première commande permet d'aller dans le répertoire où se trouve notre projet.
			La seconde commande va télécharger \textit{boost} dans le projet, cette étape peut être longue, comme la suivante\footnote{C'est le moment d'aller chercher un café, si vous en n'avez plus.}.
			Ensuite on va \textit{preparer} \textit{boost} pour l'utiliser. C'est-à-dire qu'on va compiler les outils nécessaires.
			La dernière commande fait appelle à l'outils \textit{make} pour compiler notre projet. En réalité, le \textit{makefile}
			se contente d'appeler l'outil de compilation de \textit{boost}.
			}
			\paragraph{}{
			Si toutes ces étapes se sont bien passées, vous êtes maintenant en mesure d'utiliser Octochat.
			}
		
		\subsection{Utilisation}
			\paragraph{}{
			Pour lancer l'application taper simplement \verb|./octochat|
			}
		
		\subsection{Sous le capot}
		
		\newpage
		
	\section{Patrons de conception}
			% diagramme UML 
			% parapgraphes qui explique le pourquoi du comment
		\subsection{Observer}
		\subsection{Factory method}
		\subsection{Interpretor}
	
	\newpage
	
	\section*{Conclusion}
		\paragraph{}{je conclu}
		
\end{document}
