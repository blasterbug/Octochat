\documentclass[a4paper]{article}
\usepackage[utf8]{inputenc}
\usepackage[frenchb]{babel}
\usepackage{ifpdf}
\usepackage{hyperref}
\title{Octochat \\ \textit{Discuter sur le réseau local sans serveur}}
\author{Alexis \textsc{Giraudet}, Benjamin \textsc{Sientzoff}}
\date{\today}
\ifpdf
\hypersetup{
    pdfauthor={Alexis Giraudet, Benjamin Sientzoff},
    pdftitle={Octochat - Discuter sur le réseau local sans serveur},
}
\fi
\begin{document}
	% page de garde avec sommaire
	\maketitle
	\vspace{5cm}
	\tableofcontents
	\newpage % passer à la page suivante
	
	\section*{Introduction}
	% présenter le cadre du projet
	% présenter le prjet en lui-même, idées de base, inspiration
	
	
	\section{titre de la section}
	
		\subsection{Présentation génrale}
		% présentation des principales classes, méthodes
		% à quoi elles servent?
		\paragraph{titre du paragraphe}{contenu}
		\paragraph{}{paragraphe sans titre}
		
		\subsection{Premier pattern}
		% l'observeur
		\paragraph{titre du paragraphe}{contenu}
		\paragraph{}{paragraphe sans titre}
		
		\subsection{Second pattern}
		% le décorateur
		\paragraph{titre du paragraphe}{contenu}
		\paragraph{}{paragraphe sans titre}
		
		\subsection{Troisième pattern}
		% toto
		\paragraph{titre du paragraphe}{contenu}
		\paragraph{}{paragraphe sans titre}
	
	
	\section*{Conclusion}
		\paragraph{}{je conclu}
		
\end{document}
